\documentclass{article}
\usepackage[utf8]{inputenc}
\usepackage{amsmath}
\usepackage{graphicx}

\title{Flujo electrónico en grafeno con varias impurezas circulares usando el RLBM (Mendoza \textit{et al.})}
\author{Julio César Pérez Pedraza}
\date{Junio de 2021}

\begin{document}

\maketitle

\section{Nanoribbon}
\subsection{$N_x = 2 N_y$}
\subsection{$N_x = 5 N_y$}
\subsection{$N_x = 10 N_y$}

\section{Impurezas consecutivas}
\subsection{2 impurezas con $r=25$}
\subsection{3 impurezas con $r=25$}
\subsection{4 impurezas con $r=25$}
\subsection{5 impurezas con $r=25$}
\subsection{2 impurezas con $r=15$}
\subsection{5 impurezas con $r=15$}
\subsection{3 impurezas con radios $r=25$, $r=15$ y $r=10$}
\subsection{3 impurezas con radios $r=25$, $r=10$ y $r=15$}
\subsection{3 impurezas con radios $r=15$, $r=10$ y $r=25$}
\subsection{3 impurezas con radios $r=10$, $r=15$ y $r=25$}
\subsection{5 impurezas con radios aleatorios (1)}
\subsection{5 impurezas con radios aleatorios (2)}

\section{Impurezas desordenadas}
\subsection{2 impurezas con radios y posiciones aleatorias}
\subsection{3 impurezas con radios y posiciones aleatorias}
\subsection{4 impurezas con radios y posiciones aleatorias}
\subsection{5 impurezas con radios y posiciones aleatorias}

\section{Porcentaje de impurezas con radios y ubicaciones aleatorias}
\subsection{$1\% $ de impurezas}
\subsection{$2\% $ de impurezas}
\subsection{$5\% $ de impurezas}
\subsection{$10\% $ de impurezas}

\end{document}